%% Author_tex.tex
%% V1.0
%% 2012/13/12
%% developed by Techset
%%
%% This file describes the coding for rsproca.cls

\documentclass[]{rsos}%%%%where rsos is the template name

%%%% *** Do not adjust lengths that control margins, column widths, etc. ***


%%%%%%%%%%% Defining Enunciations  %%%%%%%%%%%
\newtheorem{theorem}{\bf Theorem}[section]
\newtheorem{condition}{\bf Condition}[section]
\newtheorem{corollary}{\bf Corollary}[section]
%%%%%%%%%%%%%%%%%%%%%%%%%%%%%%%%%%%%%%%%%%%%%%%


\begin{document}

%%%% Article title to be placed here
\title{Performance in sensing is more important than locomotion in predator evasion by zebrafish}

\author{%%%% Author details
Arjun Nair, Christy Nguyen, and Matthew J. McHenry}

%%%%%%%%% Insert author address here
\address{Department of Ecology and Evolutionary Biology\\
University of California, Irvine\\
321 Steinhaus Hall\\
Irvine, CA 92697}

%%%% Subject entries to be placed here %%%%
\subject{Locomotion, strategy, sensing}

%%%% Keyword entries to be placed here %%%%
\keywords{xxxx, xxxx, xxxx}

%%%% Insert corresponding author and its email address}
\corres{Matthew J. McHenry\\
\email{mmchenry@uci.edu}}

%%%% Abstract text to be placed here %%%%%%%%%%%%
\begin{abstract}
The abstract text goes here. The abstract text goes here. The abstract text goes here. The abstract text goes here.
The abstract text goes here. The abstract text goes here. The abstract text goes here. The abstract text goes here.
\end{abstract}
%%%%%%%%%%%%%%%%%%%%%%%%%%%

%%%%%%%%%%%%%%% End of first page %%%%%%%%%%%%%%%%%%%%%

\maketitle

%%%%%%%%%% Insert the texts which can accomdate on firstpage in the tag "fmtext" %%%%%

%\begin{fmtext}



\section{Introduction}
%%%% Insert A head here

This demo file is intended to serve as a ``starter file''
for rsproca journal papers produced under \LaTeX\ using
rsproca.cls v1.5e.


Here is Matt's sample citation: \cite{Alexander:1974ba}

The citations are stored in the ref.bib file.  You can copy and paste citations from Papers as a "BibTex Record"

\section{Material and methods}

\subsection{Animal husbandry}
%%%% Insert B head here
All experiments were conducted on zebrafish (Danio rerio, Hamilton 1922), where larvae (5-7 days post fertilization, dpf) were preyed upon by older fish of the same species. 
To examine how these interactions vary with the size of the predator, we performed one set of experiments using adult ($\geq 9$ months old, 0.034 m +/- 0.005 m) predators and another using juveniles  ($3-4$ months old, 0.02 m +/-  0.004 m). 
% * <mmchenry@uci.edu> 2016-05-16T22:19:20.867Z:
%
% > ($>9$ months old) predators and another using juvenile predators (3-4 months old). 
%
% You'll want to also provide body lengths here.
%
% ^.
All fish were bred from wild-type (AB line) colonies housed in a flow-through tank system (Aquatic Habitats, Apopka, FL, USA) that was maintained at $\SI{28.5}{\celsius}$ on a 14:10 h light:dark cycle. 
To culture larvae, the fertilized eggs from randomized mating were cultured according to standard techniques (Westerfield, 1993).

\subsection{Live predator-prey observations}
Predation experiments were performed by placing both predator and prey in hemispherical aquarium and recording swimming with three high-speed video cameras. The walls of the aquarium (8.5 cm in diameter) were comprise of white acrylic, which served as a transluent diffuser for the IR lamps Recordings of predator-prey interactions were taken using a large scaffold-like structure. This structure was custom-built with an extruded aluminum (80/20 Inc., Columbia City, IN, USA) and housed three high-speed cameras (Precision Photron Inc., San Diego, CA, USA), a 8.5 cm diameter, semi-translucent theater dome, and three infrared (IR) LED panels (somewhere else) (Fig. 1A). % * <mmchenry@uci.edu> 2016-06-06T11:38:53.425Z:
%
% ^.
The theater dome acted as the arena for the larval and predator (adult or juvenile) zebrafish with water from the zebrafish tank system . 
Three high-speed cameras were mounted above the theater dome to record the predatory event from three different perspectives, which was necessary for the three-dimensional positional data collection. IR LED panels provided illumination for the the high-speed recordings. 
IR panels were used instead of traditional light sources because zebrafish can not see infrared light (CITE). Therefore, the IR panels provided the necessary, high-intensity lighting for the recording, but would not blind the fish in the theater dome.
% * <mmchenry@uci.edu> 2016-05-16T23:32:24.673Z:
%
% > 80-20 company
%
% Don't forget to include the location of all vendors.
%
% ^.

Predator-prey interactions were recorded between an adult or juvenile zebrafish and a larval zebrafish. One larval zebrafish and one predator zebrafish were placed in the theater dome with a partition between them. Both fish were left to acclimate to their surroundings for 15 minutes before the partition was removed. Immediately after removing the partition, all three cameras were triggered to record at the same time using a TTL (Transistor-Transistor logic) pulse and were recording synchronously at a 1000 frames per second using the native recording software (PhotronFASTCAM VIEWER, Precision Photron Inc., San Diego, CA, USA ). Recordings start about 0.5 seconds before the first predatory strike and ended about 0.5 seconds after the larval zebrafish was captured. Recordings were only collected if the predator zebrafish actively pursued and captured the larval zebrafish. To promote feeding behavior, predator zebrafish were fasted for one to two weeks.

\subsection{Data collection and statistical analysis}
Before any predator-prey recordings were captured, the experimental set-up was calibrated for three-dimensional data collection. A custom calibration body was created  and placed into the water-filled theater dome. The calibration body has 48 points suspended in three-dimensional space with known relative position. One image of the calibration body from each camera was taken and processed using “Digitizing Tools” (Hedrick, 2008) in MATLAB (2015a, MathWorks, Natick, MA, USA) to produce the three-dimensional calibration for data collection.

Each recording was digitized to collect three-dimensional midpoints on both the predator zebrafish and larval zebrafish at key moments through the predator-prey interaction. Throughout the recording, the predator zebrafish attempts to capture the larval zebrafish multiple times. For each of these capture attempts, the predator fish approaches and strikes the larval zebrafish, while the larva will attempt to escape with an escape response. The midpoint position of the predator fish and larval fish were acquired for when the predator fish starts to approach the larval zebrafish, when the predator fish started and ended suction feeding, and when the larval zebrafish initiated an escape response.

\subsection{Probabilistic modeling}
The probabilistic model simulates a virtual predator agent and prey agent engaged in a predator-prey interaction (Fig 1B). Each agent have two principal variables that change over the span of a simulation: position and orientation (in two dimensions). Each agent transitions between two states, which will dictate how the agent’s state variables change. How the agents transition between states and how the principal variables evolve depends on the thirteen experimentally derived parameters of the model. Five of the thirteen parameters are defined by lognormal probability distributions. Different sets of parameters values and probability distributions were used to simulate predator-prey interactions between adult and larval zebrafish, and juvenile and larval zebrafish.

Parameters that are defined by a probability distribution takes on a random value every time the model evaluates the parameter in the simulation. The value for these specific parameters is randomly sampled from its associated probability distribution. Though these probability distribution parameters are being sampled from a boundless distribution, the model enforces upper and lower bounds of valid random values. These bounds are determined by the minimum and maximum values observed experimentally. 

\begin{table}[!h]
\scriptsize
\caption{Table of parameters: Adult-larval zebrafish interactions}%%%Table caption goes here
\label{table_example}
\begin{tabular}{lllll}%%%The number of columns has to be defined here
\hline
Variable name &Type &Sample size &value(s) &units \\
\hline
\textit{Predator:}& & & & \\
Strike distance &Probability distribution &77 &$\mu$ = -4.97996, $\sigma$ = 0.447846 &meters \\
Strike duration &Probability distribution &77 &$\mu$ = -3.16594, $\sigma$ = 0.330798 &seconds \\
Approach speed &Constant &N/A &0.13 &meters/second \\
Predator delay &Constant &N/A &0.01 &Seconds \\
Capture probability &Probability function &77 &\textit{r} = 573.8207, \textit{$x_0$} = 0.0052  &probability \\
\textit{Prey:}& & & & \\
Escape distance &Probability distribution &77 &$\mu$ = -4.54594, $\sigma$ = 0.587484 &meters \\
Escape angle &Probability distribution &206 &$\mu$ = 0.1442, $\sigma$ = 0.449289 &radian \\
Escape duration &Probability distribution &77 &$\mu$ = -1.36936, $\sigma$ = 0.552332 &seconds \\
Escape direction &Constant &206 &0.6959 &Probability \\
Escape latency &Constant &15 &0.008 &seconds \\
Escape speed &Constant &12 &0.4 &meters/second \\\hline
\end{tabular}
\end{table}%%%End of the table

\begin{table}[!h]
\scriptsize
\caption{Table of parameters: Juvenille-larval zebrafish interactions}%%%Table caption goes here
\label{table_example}
\begin{tabular}{lllll}%%%The number of columns has to be defined here
\hline
Variable name &Type &Sample size &value(s) &units \\
\hline
\textit{Predator:}& & & & \\
Strike distance &Probability distribution &91 &$\mu$ = -5.09997, $\sigma$ = 0.647578 &meters \\
Strike duration &Probability distribution &91 &$\mu$ = -3.20767, $\sigma$ = 0.399461 &seconds \\
Approach speed &Constant &N/A &0.05 &meters/second \\
Predator delay &Constant &N/A &0.01 &Seconds \\
Capture probability &Probability function &91 &\textit{r} = 1990.4, \textit{$x_0$} = 0.0016  &probability \\
\textit{Prey:}& & & & \\
Escape distance &Probability distribution &91 &$\mu$ = -4.94093, $\sigma$ = 0.581919 &meters \\
Escape angle &Probability distribution &206 &$\mu$ = 0.1442, $\sigma$ = 0.449289 &radian \\
Escape duration &Probability distribution &91 &$\mu$ = -1.16678, $\sigma$ = 0.523393 &seconds \\
Escape direction &Constant &206 &0.6959 &Probability \\
Escape latency &Constant &15 &0.008 &seconds \\
Escape speed &Constant &12 &0.4 &meters/second \\\hline

\end{tabular}
\end{table}%%%End of the table

Most parameters for the model were derived from the digitized data. Most constant parameters were averaged values derived from the digitized data of the live predator-prey interactions. Escape direction and angle were derived from data from an unpublished study. Escape latency was determined in another study as well (Nair 2015). Probability distributions for parameters were derived using the MATLAB distribution fitting application in the statistics toolbox (MATLAB etc.). All experimental measurements for each parameter were pooled together and fitted with a lognormal probability distribution (Eq. 2.1).

\begin{align}\label{1.1} %%% Equation lognormal distribution
\begin{split}
P(x) = \frac{1}{x\sigma \sqrt{2 \pi}} e^-{\frac{(ln(x)-\mu)^2}{2\sigma ^2}}
\end{split}
\end{align}

The capture probability function was derived using experimental data. For each capture attempt observed by the predator fish in the live predator-prey experiment, it was determined whether the capture was success or failure and what the distance was between the predator fish and larval zebrafish midway through suction feeding. The successful capture data and the failed capture data were independently binned by its associated distance value. Each success and fail bin that matched up in its distance range was used to create a capture success probability (success count/(success count + fail count)). Then a sigmoid function was fitted to the series of capture success probabilities (Eq. 2.2).

\begin{align}\label{1.2} %%%Equation for sigmoid function
\begin{split}
Capture\,probability = \frac{1}{1+e^{-r(distance-x_0)}}
\end{split}
\end{align}

Some parameter values were derived by matching simulation results to experimental results. Predator delay was a parameter that could not be measured from our recordings and approach speed of the predator fish was quite variable. Therefore, values for predator delay and approach speed were determined post hoc. Simulations with varying values of predator delay and predator speed were ran and compared to experimental results (based off the methodology in the validation section). The parameter set values that yielded the greatest similarity (determined by a Kolmogorov-Smirnov test) was used for predator delay and speed.

The probabilistic model initializes every simulation randomly. The predator agent’s starting position is always at the origin of the two-dimensional virtual arena. Conversely, the prey agent’s starting position is randomly place within 8.5 cm of the origin. This is to force the starting conditions to be similar to the live predator-prey experiments. However, once the simulation has started, the predator and prey agents are not confined to a limited space like in the live predator-prey experiments. The prey agent\textsc{\char39}s orientation is randomized at the beginning of the simulation, while the predator agent always starts facing the prey agent.

The predator agent transitions between two states: tracking state and striking state. The simulation starts with the predator agent in the tracking state. In this state, the predator agent is simply moving closer to the prey agent, with a speed equal to the approach speed parameter. The predator agent will always take the shortest path to the prey agent. If the prey agent is moving, the predator agent will update its orientation to be always facing the prey agent. How often the predator agent updates its orientation depends on the predator delay parameter. The predator delay parameter represents the sensory and motor time delays present in the predator when reacting to a moving prey. The predator agent will transition into the striking state if it reaches a threshold distance between itself and the prey agent. The value of the threshold distance is determined by the strike distance parameter. Since strike distance is a probability distribution parameter, every time a predator agent is in the tracking state, the escape distance parameters will have a different random value based on the associated probability distribution. Once in the striking state, the predator agent will not adjust its course and swims straight for a fixed amount of time. In this state, the predator attempts to capture the prey agent. The amount of time the predator stays in the striking state is determined by the strike duration parameter, another probability distribution parameter. Once the striking state is finished, if the simulation isn’t over, the predator agent will transition back into the tracking state.

The prey agent also transitions between two states throughout the simulation: waiting state and escaping state. The prey agent starts the simulation in the waiting state. In the waiting state, the prey agent is idle and doesn’t move, mimcking larval zebrafish behavior. The prey agent will transition to the escaping state when the predator comes within a certain threshold distance determined by the escape distance parameter, another probability distribution parameter. Once the escape distance is met, the prey agent will transition into the escaping state where it starts locomoting with a speed determined by the escape speed parameter. The escaping state lasts for a certain duration depending on the escape duration parameter, a probability distribution parameter. For a short period at the start of the state, the prey agent will not move; this represents the delay in neural activation before actual center of mass movement. The length of the delay period is determined by the escape latency parameter. Once passed the escape latency period, the prey agent must pick which side of of its body it decides to escape towards. Previous studies show larval zebrafish have a bimodal escape response, where larval zebrafish are more likely to escape towards the side of their body facing away from the predator (Bill paper and unpublished study). Escape direction parameters is the probability of larval zebrafish escaping towards the away-facing side of their body. The angle at which they escape with on the given side of their body is determined by a probability distribution parameter, escape angle. The prey agent escapes following a straight line path, using a triangle shaped velocity curve. Throughout the escaping state, the velocity of the prey agent follows a custom triangle pulse that peaks 20\% into the duration of the escaping state. The magnitude of the the pulse is determined by the escape speed parameter. Once the escaping state is finished, the prey transitions back to the waiting state.

The simulation ends when the prey agent is captured or when the prey has escaped twenty times. One simulation termination condition is when the predator agent is able to capture the prey agent during the strike state. Half way during the striking state, the simulation evaluates whether the predator agent has successfully capture the prey agent by measuring the distance between agents halfway through the strike state and evaluating the capture probability function with this distance to get the probability of a successful capture. The model will randomly generate a number between zero and one and compare it to the resultant success probability. If the randomly generated number is equal to or lower than the resultant success probability, the prey is captured and the simulation ends. The model has a built-in hard cutoff that ends the simulation if the prey escape twenty times. This ensures simulations do not run indefinitely.

\subsection{Model validation}
Simulation results and experimental results were compared using a Kolmogorov–Smirnov test. This test was chosen over a kruskal-wallis test because the Kolmogorov-Smirnov test emphasizes differences in the shape of the distribution, rather than differences in the median. We believed this was a better approach to compare distributions in our study. The number of successful prey escapes before capture for each experiment and for a thousand simulations of the model were compiled into two groups, experimental and simulation, and were compared using a Kolmogorov-Smirnov test.

\subsection{Sensitivity analysis}
The model’s parameter set was systematically varied to yield different escape probabilities for the prey agent. For parameters of the prey agent, the mean of the parameter (or the value itself if the parameter is a constant) was varied between -90\% and 100\% of its original value by increments of 10\%. For parameters that have an associated probability distribution, the $\mu$ parameter of the lognormal distributions are adjusted to create a change in distribution’s mean. The upper and lower sampling bounds are also shifted according. A thousand simulations were ran for each incremental change in the mean of a given parameter and  the number of successful escapes before capture was collected  for each simulation. Comparisons between increment changes in parameters were made using a Kruskal-Wallis test. Data for each increment change in the mean was represented as an escape probability. This was calculated by taking the the total sum of successful escapes and dividing it by the total attempted escapes.

\section{Results}
Data collected from live predator prey experiments alone could not discern performance differences between locomotion and sensing. For the live experiments, a suite of variables pertaining to locomotion and sensing for the prey (larval zebrafish) were collected from the predator-prey recordings (Table 2.1-2). Differences between successful escapes and failed escapes in each parameter were insignificant (Mann-Whitney U test: P > 0.05, N = 77 with adult zebrafish, 91 with juvenille zebrafish). Furthermore, no predator zebrafish (adult and juvenile) collected parameter was indicative whether a larval zebrafish could escape a predatory strike (Mann-Whitney U test: P > 0.05, N = 77, 91). This suggests that neither locomotion or sensing had an significant impact on whether a prey escape attempt was successful or not. This seemed unreasonable since it well known both sensory and motor systems are important for prey survival (CITE). We suspect this discrepancy emerges from the fact that the behavior of predator zebrafish and larval zebrafish are modified by each other. This leads to a coupled system of behavior where there are not strict measurable controls. Therefore, we decided to computationally model the observed predator-prey interactions, allowing the ability to control for various aspects of the predatory encounter.

The probabilistic model confidently mimics the live predator-prey experiments. The probabilistic model uses experimentally derived parameters to simulate a predator-prey interaction between a predator agent and a prey agent. Both of the agents behave similarly to their biological counterparts (Fig. 2A). To quantitatively compare the model and the predator-prey experiments, we collected the number of successful escapes  larval zebrafish and the prey agent make before being captured and compared for significant differences. The comparison revealed that the two datasets are quantitatively similar (Kolmogorov-Smirnov test: P >> 0.05, N = 77 for experiments,  N = 1000 for simulations). A similar result is seen with the juvenile-larval zebrafish datasets (Kolmogorov-Smirnov test: P >> 0.05, N = 91 for experiments,  N = 1000 for simulations).Given the probabilistic model can replicate our predator-prey experiments, we conducted the rest of our analysis just using the probabilistic model. Furthermore, many of our results observed with yoru adult/larval zebrafish parameter set were mirrored by the juvenile/larval zebrafish parameter set. Therefore, this paper will focus on the results of the adult/larval zebrafish probabilistic model.

A sensitivity analysis of the prey parameters revealed that escape speed and escape distance were the most impactful parameters for prey survival. we conducted a sensitivity analysis where we modulated each prey parameter. For each parameter, the mean value of the parameter was changed by increments of 10\%. For each incremental change in the parameter, we observed a change in the escape probability, the chance a prey agent will escape for any given predatory strike (Fig. 3). Changes in escape duration, escape direction, and escape angle lead to minimal changes in escape probability. However, escape distance and escape speed had much more substantial effects on escape probability compared to the rest of the prey parameters. By increasing the mean value of the escape distance, escape probability increases significantly. However, decreasing the mean value of the escape distance creates greater, significant decreases to escape probability. Increasing escape speed did not significantly increase in escape probability. Significant decreases to escape probability only occurred when escape speed was reduced by 50\% or more of its original value.

There are compensatory effects for escape probability between escape speed and escape distance. Since escape speed and escape distance were the two parameters that had the greatest impact on escape probability of the prey agent, we conducted a two-dimensional sensitivity analysis where escape distance and escape speed were both incrementally changed (Fig. 5A). By increasing and decreasing both parameters, escape probability increases and decreases respectively. However, escape speed and escape distance affect each other in different manners (Fig. 5B,C). Reflecting the patterns seen in the one-dimensional sensitivity analysis, escape distance can modulate the patterns of escape speed has on escape probability (Fig. 5B). By keeping escape speed constant, increasing escape distance can slightly increase the overall escape probability. However, similar to the one-dimensional analysis, decreasing escape distance while escape speed is constant can create large, significant decreases to escape probability. Also similar to the one-dimensional analysis, decreasing escape speed is the onyl way to observed significant changes in escape probability when escape distance is held constant (Fig. 5C).  While escape distance is held constant, increasing escape speed did not substantially increase escape probability. This exemplifies the "binary" effect escape speed had on escape probability .

\section{Discussion}


\section*{Sample equations}

Sample equations.

%%% Numbered equation
\begin{align}\label{1.1}
\begin{split}
\frac{\partial u(t,x)}{\partial t} &= Au(t,x) \left(1-\frac{u(t,x)}{K}\right)-B\frac{u(t-\tau,x) w(t,x)}{1+Eu(t-\tau,x)},\\
\frac{\partial w(t,x)}{\partial t} &=\delta \frac{\partial^2w(t,x)}{\partial x^2}-Cw(t,x)+D\frac{u(t-\tau,x)w(t,x)}{1+Eu(t-\tau,x)},
\end{split}
\end{align}

\begin{align}\label{1.2}
\begin{split}
\frac{dU}{dt} &=\alpha U(t)(\gamma -U(t))-\frac{U(t-\tau)W(t)}{1+U(t-\tau)},\\
\frac{dW}{dt} &=-W(t)+\beta\frac{U(t-\tau)W(t)}{1+U(t-\tau)}.
\end{split}
\end{align}

%%%% Unnumbered equation
\begin{eqnarray}
\frac{\partial(F_1,F_2)}{\partial(c,\omega)}_{(c_0,\omega_0)} = \left|
\begin{array}{ll}
\frac{\partial F_1}{\partial c} &\frac{\partial F_1}{\partial \omega} \\\noalign{\vskip3pt}
\frac{\partial F_2}{\partial c}&\frac{\partial F_2}{\partial \omega}
\end{array}\right|_{(c_0,\omega_0)}\notag\\
=-4c_0q\omega_0 -4c_0\omega_0p^2 =-4c_0\omega_0(q+p^2)>0.
\end{eqnarray}

%\end{fmtext}



\section*{Data accessibility}

\section*{Authors' contributions}

\section*{Competing interests}
We declare we have no competing interests.

\section*{Funding}
Insert the Acknowledgment text here.

\section*{Acknowledgments}
Insert the Acknowledgment text here.


%%%%%%%%%% Insert bibliography here %%%%%%%%%%%%%%
%\section*{References}

\bibliography{ref}
\bibliographystyle{prsb}   %References the PRSB style file

\section*{Figures \& Tables}

The output for figure is:

\begin{figure}[!h]
%\centering\includegraphics[width=2.5in]{xxxxxx.eps}
%%% where xxxxxx name represents "figurename.eps"
\caption{Insert figure caption here}
\label{fig_sim}
\end{figure}

\vspace*{-10pt}

\noindent The output for table is:

\begin{table}[!h]
\caption{An Example of a Table}%%%Table caption goes here
\label{table_example}
\begin{tabular}{llll}%%%The number of columns has to be defined here
\hline
date &Dutch policy &date &European policy \\
\hline
1988 &Memorandum Prevention &1985 &European Directive (85/339) \\
1991--1997 &{\bf Packaging Covenant I} & & \\
1994 &Law Environmental Management &1994 &European Directive (94/62) \\
1997 &Agreement Packaging and Packaging Waste & & \\
1998--2002 &{\bf Packaging Covenant II} & & \\
2003--2005 &{\bf Packaging Covenant III} & & \\
2006--2007 &{\bf Decree on Packaging and paper} & & \\\hline
\end{tabular}
\end{table}%%%End of the table


\end{document}